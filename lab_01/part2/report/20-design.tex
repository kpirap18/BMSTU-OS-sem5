\chapter{Конструкторская часть}
В данном разделе представлены схемы муравьиного алгоритма и алгоритма полного перебора для решения задачи коммивояжера.

\section{Требования к вводу}
Ниже описанные требования необходимо соблюдать для верной работы программы.

\begin{enumerate}
	\item При запуске программы в командной строке необходимо прописать 1 или 0 (это является выбором режима, 0 -- работа с пользователем, 1 -- режим тестирования, записи в лог файл);
	\item Если на вход в командной программе подается 1, то необходимо также ввести имя файла с данными.
\end{enumerate}

\section{Разработка алгоритмов}

\subsection{Алгоритм полного перебора}
Схема алгоритма полного перебора для решения задачи коммивояжера представлена на рисунке \ref{img:s1}. 

\img{200mm}{s1}{Схема алгоритма полного перебора}

\newpage

Схема алгоритма нахождения всех перестановок в графе, использующаяся в алгоритме полного перебора, предоставлена на рисунке \ref{img:s2}.

\img{200mm}{s2}{Схема алгоритма нахождения всех перестановок в графе}

\newpage

\subsection{Муравьиный алгоритм}

Схема реализации муравьиного алгоритма для решения задачи коммивояжера представлена на рисунке \ref{img:s3}.
Необходимые параметры -- матрица феромонов, значение минимального пути.

\img{200mm}{s3}{Схема реализации муравьиного алгоритма}

\newpage

Схема алгоритма выбора оптимального пути для муравья представлена на рисунке \ref{img:sss1}.

\img{200mm}{sss1}{Схема алгоритма выбора оптимального пути для муравья}

\newpage

\section{Вывод}

В данном разделе были рассмотрены схемы алгоритма полного перебора и муравьиного алгоритма.